\newcommand{\sortie}{la vitesse de rotation de la manivelle $\Omega_v(t)$ ($rad.s^{-1}$)}
\newcommand{\entree}{la tension en entrée du moteur $u_m(t)$ (V)}
\newcommand{\systequations}{\item $J.\dfrac{d \omega_m(t)}{dt}=C_m(t)$}
\newcommand{\grandeurs}{\item la vitesse de rotation du moteur $\omega_m(t)$,
\item \sortie.}
\newcommand{\fonctiontransfert}{$H(p)=\dfrac{\Omega_v(p)}{U_m(p)}$}

\newcommand{\correction}{\newpage

\pagestyle{correction}\setcounter{section}{0}

\begin{eqnarray}
J\cdot \frac{d\omega (t)}{dt}=Cm(t) \\
Um(t)=R\cdot i(t) +e(t) \\
e(t)=K_e\cdot \omega (t) \\
Cm(t)=K_t\cdot i(t)
\end{eqnarray}

$H(p)=\dfrac{\dfrac{1}{K_e}}{1+\dfrac{R\cdot J}{K_e\cdot K_t}\cdot p}$

Avec $U=\frac{U_0}{p}$.

$\Omega(p)=\dfrac{\dfrac{U_0}{K_e}}{p\cdot \left(1+\dfrac{R\cdot J}{K_e\cdot K_t}\cdot p\right)}$, donc $\omega(t)=K(1-e^{-\frac{t}{\tau}})$

\begin{eqnarray}
J\cdot \frac{d\omega (t)}{dt}=Cm(t)-Cr(t)-f.\omega(t) \\
Um(t)=R\cdot i(t) +e(t) \\
e(t)=K_e\cdot \omega (t) \\
Cm(t)=K_t\cdot i(t)
\end{eqnarray}

Avec $Cr(t)=Cr_0.cos(\omega_r.t)+Cs_0$.\\

$J\cdot p \cdot \Omega = Cm(p)-Cr(p)-f.\Omega$

$Um(p)-\dfrac{R\cdot Cr(p)}{K_t}=\left(\dfrac{R\cdot (J\cdot p+f)}{K_t}+K_e\right)\cdot\Omega=\left(\dfrac{R\cdot (J\cdot p+f)}{K_t}+K_e\right)\cdot\Omega$

$\Omega(p)=\dfrac{\dfrac{Kt}{Kt\cdot Ke+R\cdot f}}{1+\dfrac{R\cdot J}{Kt\cdot Ke+R\cdot f}\cdot p}\cdot \left(Um(p)-\dfrac{R\cdot Cr(p)}{K_t}\right)$

Avec $U=\frac{U_0}{p}$ et $Cr=0$.

$\Omega_1(p)=\dfrac{\dfrac{Kt}{Kt\cdot Ke+R\cdot f}}{1+\dfrac{R\cdot J}{Kt\cdot Ke+R\cdot f}\cdot p}\cdot \dfrac{U_0}{p}$

Avec $U=0$ et $Cr=\dfrac{Cr_0\cdot p}{p^2+\omega_r^2}+\dfrac{Cs_0}{p}$.


$\Omega_2(p)=-\dfrac{\dfrac{Kt}{Kt\cdot Ke+R\cdot f}}{1+\dfrac{R\cdot J}{Kt\cdot Ke+R\cdot f}\cdot p}\cdot \dfrac{R}{K_t}\cdot \left(\dfrac{Cr_0\cdot p}{p^2+\omega_r^2}+\dfrac{Cs_0}{p}\right)$

On prend:

$K=-\dfrac{Kt}{Kt\cdot Ke+R\cdot f}\cdot \dfrac{R}{K_t}$

$\tau=\dfrac{R\cdot J}{Kt\cdot Ke+R\cdot f}$

$\Omega_2(p)=\dfrac{K}{1+\tau\cdot p}\cdot \left(\dfrac{Cr_0\cdot p}{p^2+\omega_r^2}+\dfrac{Cs_0}{p}\right)=\dfrac{K\cdot(Cr_0\cdot p^2+Cs_0\cdot p^2+Cs_0\cdot \omega_r^2)}{(1+\tau\cdot p)\cdot(p^2+\omega_r^2)\cdot p}$

~\

$\Omega_2(p)=\dfrac{A}{1+\tau\cdot p}+\dfrac{B\cdot p+C}{p^2+\omega_r^2}+\dfrac{D}{p}$

$=\dfrac{A\cdot p^3+A\cdot \omega_r^2\cdot p+B\cdot p^2+C\cdot p+B\cdot \tau\cdot p^3+C\cdot \tau \cdot p^2+D\cdot p^2+D\cdot \omega_r^2+D\cdot \tau\cdot p^3+D\cdot \omega_r^2\cdot \tau\cdot p}{(1+\tau\cdot p)\cdot(p^2+\omega_r^2)\cdot p}$

\begin{eqnarray}
A+B\cdot \tau+D\cdot \tau=0 \\
B+C\cdot \tau+D=K\cdot(Cr_0+Cs_0) \\
A\cdot \omega_r^2+C+D\cdot \omega_r^2\cdot \tau=0 \\
D \cdot \omega_r^2=K\cdot Cs_0\cdot \omega_r^2
\end{eqnarray}

$D=K\cdot Cs_0$

$C=B\cdot \tau \cdot \omega_r^2$

$B\cdot (1+\tau^2 \cdot \omega_r^4)+D=K\cdot(Cr_0+Cs_0)$

$B=\dfrac{K\cdot Cr_0}{1+\tau^2 \cdot \omega_r^4}$

$C=\dfrac{K\cdot Cr_0\cdot \tau \cdot \omega_r^2}{1+\tau^2 \cdot \omega_r^4}$

$A=-\dfrac{K\cdot Cr_0\cdot \tau}{1+\tau^2 \cdot \omega_r^4}-\tau\cdot K\cdot Cs_0$


$\omega_2(t)=\dfrac{A}{\tau}.e^{\frac{-t}{\tau}}+B\cdot cos(\omega_r\cdot t)+\dfrac{C}{\omega_r}\cdot sin(\omega_r\cdot t)+D$
}