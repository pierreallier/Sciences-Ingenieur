\input{../../headers/tpheaders_2.tex}
%\input{"\dossierilot/\sequence-\type\num-I\ilotstr.tex"}
\newcommand{\sortie}{la tension de la corde $Fc(t)$ (N)}
\newcommand{\entree}{la tension en entrée du moteur $u(t)$ (V)}
\newcommand{\systequations}{$J.\dfrac{d \omega_m(t)}{dt}=C_m(t)-r.R_p.F_c(t)$}
\newcommand{\grandeurs}{\item la vitesse de rotation du moteur $\omega_m(t)$,
\item la vitesse de déplacement du charriot $v_c(t)$,
\item l'allongement de la corde $\delta_c(t)$.}
\newcommand{\fonctiontransfert}{$H(p)=\dfrac{F_c(p)}{U_m(p)}$}



\prob{Modéliser un Système Linéaire Continu et Invariant} \\

\graphicspath{{../../img/}}
\begin{center}
\def\svgwidth{\columnwidth}
\input{"../../img/triptyque.pdf_tex"}
\end{center}

La démarche de l’ingénieur permet :
\begin{itemize}
 \item De vérifier les performances attendues d’un système, par évaluation de l’écart entre un cahier des charges et les réponses expérimentales (écart 1),
 \item De proposer et de valider des modèles d’un système à partir d’essais, par évaluation de l’écart entre les performances mesurées et les performances simulées (écart 2),
 \item De prévoir le comportement à partir de modélisations, par l’évaluation de l’écart entre les performances simulées et les performances attendues du cahier des charges (écart 3).
\end{itemize}

\documentsressource{Pour ce TP, vous aurez besoin:
\begin{itemize}
 %\item du \href{\miseneoeuvresysteme}{document} de mise en \oe uvre du système,
 \item de la procédure d'utilisation de Simscape disponible à la page \pageref{proceduresimscape},
\end{itemize}}

\newpage


\ifdef{\public}{\cleardoublepage}{}

\section{Modéliser la chaîne d'énergie du système en Boucle Ouverte}

Nous allons dans cette partie chercher à déterminer la fonction de transfert qui lie \entree et \sortie.

On donnera le système d'équations suivants: \\
\systequations

Les caractéristiques du système sont données sur le SITE INTERNET. Si les valeurs numériques de certaines grandeurs ne sont pas données, c'est qu'elles seront négligées par la suite.

\paragraph{Question 1:} Donner l'ensemble des équations temporelles permettant de modéliser un moteur à courant continu.

\paragraph{Question 2:} Déterminer les équation qui lient les paramètres suivants:
\begin{itemize}
 \grandeurs
\end{itemize}

\paragraph{Question 3:} Passer ces équations dans le domaine de Laplace.

\paragraph{Question 8:} Mettre le système sous la forme de la fonction de transfert suivante: \fonctiontransfert. Donner les caractéristiques de cette fonction de transfert et vérifier l'homogénéité des constantes déterminées.

\titleres{Simulation du comportement du modèle}

Le logiciel \textbf{Scilab} permet de tracer la réponse temporelle d'une fonction de transfert donnée.

Pour cela, il suffit de lancer le logiciel et d'aller dans le module \textbf{Xcos}.

Dans le dossier \textbf{CPGE} du navigateur de palettes, vous trouverez, par exemple:
\begin{itemize}
 \item une \textit{entrée}: STEP\_FUNCTION,
 \item un \textit{Opérateur linéaire}: CLR, vous modifierez sa fonction de transfert afin d'obtenir ce que vous souhaitez observer,
 \item une \textit{sortie}: SCOPE,
 \item un \textit{outil d'analyse}: REP\_TEMP.
\end{itemize}

~\

Faire glisser ces blocs sur une page vierge du module xcos et cliquer sur la flèche permettant de lancer la simulation.

\paragraph{Question 9:} Effectuer le tracé de la fonction de transfert vue en TD afin d'apprendre à maitriser l'outil Scilab.

\paragraph{Question 10:} Modifier la fonction de transfert afin d'y insérer celle modélisée dans la section modélisation.

\paragraph{Question 11:} Critiquer les résultats obtenus et analyser le lien entre les tracés obtenus par simulation et ceux obtenus durant l'expérimentation.

~\

L'ensemble des réponses que vous aurez donné dans cette partie devra être utilisé afin de compléter le document de présentation. Les diapositives pourront être agrémentées comme vous le souhaitez.

\ifdef{\public}{\end{document}}{}

\newpage

\pagestyle{correction}\setcounter{section}{0}

Modélisation
\begin{center}
$H(p)=\dfrac{F_c(p)}{U_m(p)}=\dfrac{\dfrac{K_m}{R_m.R_p.r}}{1+\dfrac{K_e.K_m}{R_m.K_c.R_p^2.r^2}.p+\dfrac{R_m.J}{R_m.K_c.R_p^2.r^2}.p^2}$
\end{center}

\end{document}