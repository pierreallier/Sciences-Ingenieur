\input{../../headers/tpheaders_2.tex}
\input{"\dossierilot/\sequence-\type\num-I\ilotstr.tex"}


\prob{Modéliser un Système Linéaire Continu et Invariant} \\

\graphicspath{{../../img/}}
\begin{center}
\def\svgwidth{\columnwidth}
\input{"../../img/triptyque.pdf_tex"}
\end{center}

La démarche de l’ingénieur permet :
\begin{itemize}
 \item De vérifier les performances attendues d’un système, par évaluation de l’écart entre un cahier des charges et les réponses expérimentales (écart 1),
 \item De proposer et de valider des modèles d’un système à partir d’essais, par évaluation de l’écart entre les performances mesurées et les performances simulées (écart 2),
 \item De prévoir le comportement à partir de modélisations, par l’évaluation de l’écart entre les performances simulées et les performances attendues du cahier des charges (écart 3).
\end{itemize}

\documentsressource{Pour ce TP, vous aurez à votre diposition les documents suivants:
\begin{itemize}
 \item La \miseenoeuvre\ du système,
 \item de la procédure d'utilisation de Simscape disponible à la page \pageref{proceduresimscape},
 \item Les divers documents des \urlsysteme.
\end{itemize}}
\newpage

\section{Modéliser la chaîne d'énergie du système en Boucle Ouverte}

Nous allons dans cette partie chercher à déterminer la fonction de transfert qui lie \entree et \sortie.

On donnera le système d'équations suivants: \\
\begin{itemize}
 \systequations
\end{itemize}~\

Les caractéristiques du système sont données sur la page \urlsysteme. Si les valeurs numériques de certaines grandeurs ne sont pas données, c'est qu'elles seront négligées par la suite.

\question{Donner l'ensemble des équations temporelles permettant de modéliser un moteur à courant continu.}

~\

\textbf{La question suivante est une des plus complexes et plus importantes du TP.} Pour y répondre, il faudra fouiller dans le logiciel de pilotage du système, proposer des mesures à réaliser sur le système,... et proposer à votre enseignant des idées sur la démarche à suivre.

\question{Déterminer les équation qui lient les paramètres suivants:
\begin{itemize}
 \grandeurs
\end{itemize}}

\section{Résolution des équations du modèle}

\question{Passer ces équations dans le domaine de Laplace.}

\question{Mettre le système sous la forme de la fonction de transfert suivante: \fonctiontransfert. }

\question{Donner les caractéristiques de cette fonction de transfert et vérifier l'homogénéité des constantes déterminées.}

~\

Afin de tracer les réponses temporelles, il est possible d'utiliser un script python téléchargeable 
\href{https://github.com/Costadoat/Sciences-Ingenieur/raw/master/S02\%20Syst\%C3\%A8mes\%20Lin\%C3\%A9aires\%20Continus\%20Invariants/C01\%20Pr\%C3\%A9sentation\%20des\%20SLCI/02-C01.py}{ici}.

\section{Expérimentation sur le système réel}

Mettre en \oe uvre le système, et préparer la prise de mesure. 

\question{Après avoir analysé le logiciel d'expérimentation, déterminer quelles grandeurs peuvent être utilisées en entrée. Est-ce que \entree en fait partie ?}

\question{Relever les tracés des réponses temporelles de \entree et de \sortie.}

\question{En mettant en parallèle les tracés issus de la modélisation et ceux issus de l'expérimentation, conclure quand à la validation du modèle utilisé.}

\ifdef{\public}{\end{document}}{}

\correction

\end{document}
