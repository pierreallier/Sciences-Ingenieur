\input{../../headers/tpheaders_2.tex}
\input{"\dossierilot/\sequence-\type\num-I\ilotstr.tex"}


\prob{L'étude d'un Système Linéaire Continu et Invariant} \\

\graphicspath{{../../img/}}
\begin{center}
\def\svgwidth{\columnwidth}
\input{"../../img/triptyque.pdf_tex"}
\end{center}

La démarche de l’ingénieur permet :
\begin{itemize}
 \item De vérifier les performances attendues d’un système, par évaluation de l’écart entre un cahier des charges et les réponses expérimentales (écart 1),
 \item De proposer et de valider des modèles d’un système à partir d’essais, par évaluation de l’écart entre les performances mesurées et les performances simulées (écart 2),
 \item De prévoir le comportement à partir de modélisations, par l’évaluation de l’écart entre les performances simulées et les performances attendues du cahier des charges (écart 3).
\end{itemize}

\documentsressource{Pour ce TP, vous aurez à votre diposition les documents suivants:
\begin{itemize}
 \item La \miseenoeuvre\ du système,
 \item de la procédure d'utilisation de Simscape disponible à la page \pageref{proceduresimscape},
 \item Les divers documents des \urlsysteme.
\end{itemize}}


\newpage

\section{Sollicitation harmonique}

Mettre en \oe uvre le système en utilisant la procédure donnée (DEMANDER A L'ENSEIGNANT).

Afin de déterminer la fonction de transfert \dusysteme, par une étude harmonique, nous allons le solliciter avec une entrée sinusoïdale. Les opérations suivantes seront effectuées plusieurs fois:
\begin{enumerate}
 \item Afficher le tracé présentant la consigne d'entrée et la réponse correspondante,
 \item Mesurer l'amplitude et la fréquence du signal d'entrée, l'amplitude et la fréquence du signal de sortie ainsi que le retard de ce dernier par rapport à l'entrée,
 \item Déterminer le gain $Gdb$ (en $db$) et le déphasage $\phi$ (en $rad$) correspondant à la pulsation $\omega$ ($rad.s^{-1}$) de la sollicitation.
\end{enumerate}

\question{Effectuer les opérations précédentes plusieurs fois afin d'obtenir une première ébauche de diagramme de Bode de la fonction de transfert \dusysteme.}

\question{Déterminer approximativement l’intervalle dans lequel se trouve la valeur de la pulsation de coupure.}

\question{Effectuer de nouveau les opérations précédentes autour de la pulsation de coupure afin d’affiner le diagramme de Bode déterminé précédemment.}

\section{Identification de la fonction de transfert}

\question{A partir des résultats précédents, déterminer la forme canonique de la fonction de transfert \dusysteme.}

\question{Utiliser le script python qui permet de tracer le diagramme de Bode à partir de la fonction de transfert et le comparer au tracé déterminé expérimentalement.}

\question{Solliciter le système avec un échelon en entrée et valider la fonction de transfert trouvée précédemment.}

\ifdef{\public}{\end{document}}{}

\newpage

\pagestyle{correction}\setcounter{section}{0}

Modélisation
\begin{center}
$H(p)=\dfrac{F_c(p)}{U_m(p)}=\dfrac{\dfrac{K_m}{R_m.R_p.r}}{1+\dfrac{K_e.K_m}{R_m.K_c.R_p^2.r^2}.p+\dfrac{R_m.J}{R_m.K_c.R_p^2.r^2}.p^2}$
\end{center}

\end{document}
