\newcommand{\id}{57}
\newcommand{\nom}{La cinématique des mécanismes}
\newcommand{\sequence}{03}
\newcommand{\num}{02}
\newcommand{\type}{TP}
\newcommand{\descrip}{Lois E/S de fermeture géométrique et cinématique. Simulation du comportement de modèles. Proposer des lois de commande en fonction d'exigences. Présenter les modèles acausaux}
\newcommand{\competences}{Mod2-C10-1: Modèle de solide indéformable \\ &  Mod2-C11: Modélisation géométrique et cinématique des mouvements entre solides indéformables \\ &  Rés-C1: Loi entrée sortie géométrique et cinématique \\ &  Rés-C6: Utilisation d'un solveur ou d'un logiciel multi physique \\ &  Com1-C1: Différents descripteurs introduits dans le programme \\ &  Com2-C4: Outils de communication}
\newcommand{\nbcomp}{6}
\newcommand{\systemes}{Barrière Sympact}
\newcommand{\ilot}{2}
\newcommand{\ilotstr}{02}
\newcommand{\dossierilot}{\detokenize{Ilot_02 Barrière Sympact}}
\newcommand{\imageun}{Barriere}

\newcommand{\urlsysteme}{\href{https://www.costadoat.fr/systeme/49}{Ressources système}}
\newcommand{\solidworks}{\href{https://github.com/Costadoat/Sciences-Ingenieur/raw/master/Systemes/Barriere Sympact/Barriere_Solidworks.zip}{Modèle Solidworks}}
\newcommand{\matlabsimscape}{\href{https://github.com/Costadoat/Sciences-Ingenieur/raw/master/Systemes/Barriere Sympact/Barriere_Simscape.zip}{Modèle Simscape}}
\newcommand{\edrawings}{\href{https://github.com/Costadoat/Sciences-Ingenieur/raw/master/Systemes/Barriere Sympact/Barriere.EASM}{Modèle eDrawings}}
\newcommand{\videoavi}{\href{https://github.com/Costadoat/Sciences-Ingenieur/raw/master/Systemes/Barriere Sympact/Utilisation_barriere_1.avi}{Utilisation de la barrière}}
\newcommand{\videoavii}{\href{https://github.com/Costadoat/Sciences-Ingenieur/raw/master/Systemes/Barriere Sympact/Utilisation_barriere_2.avi}{Utilisation de la barrière avec télécommande}}
\newcommand{\experimental}{\href{https://github.com/Costadoat/Sciences-Ingenieur/raw/master/Systemes/Barriere Sympact/Barriere_experimental.zip}{Analyse de résultats expérimentaux}}
\newcommand{\schemacinematique}{Barriere_cinematique}
